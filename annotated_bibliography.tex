% This is a sample document using the University of Minnesota, Morris, Computer Science
% Senior Seminar modification of the ACM sig-alternate style to generate a simple annotated
% bibliography. The idea is that this document is fairly short, consisting of a brief description
% of your sources and how you intend to use them (or not). Most of the ``content'' of the
% generated document comes from the bibliography file, including the notes field which will
% provide the annotations.

% See https://github.com/UMM-CSci/Senior_seminar_templates for more info and to make
% suggestions and corrections.

\documentclass{sig-alternate}

\begin{document}

% --- Author Metadata here ---
%%% REMEMBER TO CHANGE THE SEMESTER AND YEAR
\conferenceinfo{UMM CSci Senior Seminar Conference, December 2013}{Morris, MN}

\title{Something about NP-complete problems}

\numberofauthors{1}

\author{
% The command \alignauthor (no curly braces needed) should
% precede each author name, affiliation/snail-mail address and
% e-mail address. Additionally, tag each line of
% affiliation/address with \affaddr, and tag the
% e-mail address with \email.
\alignauthor
Chris S. Student\\
	\affaddr{Division of Science and Mathematics}\\
	\affaddr{University of Minnesota, Morris}\\
	\affaddr{Morris, Minnesota, USA 56267}\\
	\email{cssxxxx00000@morris.umn.edu}
}

\maketitle

\begin{abstract}
This paper discusses new results in NP-complete problems and the use of distributed
networks to solve certain partial cases of NP-complete problems.
\end{abstract}

% A category with the (minimum) three required fields
\category{H.4}{Information Systems Applications}{Miscellaneous}
%A category including the fourth, optional field follows...
\category{D.2.8}{Software Engineering}{Metrics}[complexity measures, performance measures]

\terms{Delphi theory}

\keywords{ACM proceedings, \LaTeX, text tagging}

\section{Introduction}
I will focus on using the approach Blah for solving partial cases of
NP-complete problems on distributed networks.

I plan to use the following sources:
\begin{itemize}
\item I expect~\cite{OM:2008} to be one of my main sources, and I'm still looking for another two ``core'' papers to build on.
\item I may use~\cite{Brun:2008} for comparison. 
\item I'll use~\cite{Aaronson:2005, wiki:np-complete} and possibly selected chapters of~\cite{Garey:1979} as background. 
\end{itemize}

As mentioned above I need two other ``core'' papers, and I'm still looking for good examples that I can use to explain the 

I was initially considering algorithms on compete graphs as a possible topic, and looked over~\cite{winkler1984isometric, dobkin1987delaunay, folkman1970graphs} before I settled on my current topic.

% The following two commands are all you need to
% produce the bibliography for the citations in your paper.
\bibliographystyle{abbrv}
% annotated_bibliography.bib is the name of the BibTex file containing 
% all the bibliography entries for this example. Note that you *don't* include the .bib ending
% in the \bibliography command.
\bibliography{annotated_bibliography}  

% You must have a ".bib" file and remember to run:
%     pdflatex bibtex pdflatex pdflatex
% in order to see all the citation references correctly.

\end{document}



