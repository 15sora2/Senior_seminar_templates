% This is a sample document using the University of Minnesota, Morris, Computer Science
% Senior Seminar modification of the ACM sig-alternate style. Much of this content is taken
% directly from the ACM sample document illustrating the use of the sig-alternate class. Certain
% parts that we never use have been removed to simplify the example, and a few additional
% components have been added.

% See https://github.com/UMM-CSci/Senior_seminar_templates for more info and to make
% suggestions and corrections.

\documentclass{sig-alternate}
\usepackage{color}
\usepackage[colorinlistoftodos]{todonotes}

%%%%% Uncomment the following line and comment out the previous one
%%%%% to remove all comments
%%%%% NOTE: comments still occupy a line even if invisible;
%%%%% Don't write them as a separate paragraph
%\newcommand{\mycomment}[1]{}

\begin{document}

% --- Author Metadata here ---
%%% REMEMBER TO CHANGE THE SEMESTER AND YEAR AS NEEDED
\conferenceinfo{UMM CSci Senior Seminar Conference, December 2015}{Morris, MN}

\title{UMM CSci Senior Seminar LaTeX template}

\numberofauthors{1}

\author{
% The command \alignauthor (no curly braces needed) should
% precede each author name, affiliation/snail-mail address and
% e-mail address. Additionally, tag each line of
% affiliation/address with \affaddr, and tag the
% e-mail address with \email.
\alignauthor
Chris S. Student\\
	\affaddr{Division of Science and Mathematics}\\
	\affaddr{University of Minnesota, Morris}\\
	\affaddr{Morris, Minnesota, USA 56267}\\
	\email{cssxxxx00000@morris.umn.edu}
}

\maketitle
\begin{abstract}
This paper provides a sample of a \LaTeX\ document which conforms,
somewhat loosely, to the formatting guidelines for the University of
Minnesota, Morris, Computer Science Senior Seminar proceedings.
It is based heavily on (and takes material directly from) a similar document 
illustrating the format of the ACM SIG Proceedings, which we have based our
proceedings format on.

The original ACM document tried to include
\begin{quote}
every imaginable sort
of ``bells and whistles", such as a subtitle, footnotes on
title, subtitle and authors, as well as in the text, and
every optional component (e.g. Acknowledgments, Additional
Authors, Appendices), not to mention examples of
equations, theorems, tables and figures.
\end{quote}

We've removed many of the more esoteric tricks here because either
they'll never be used (e.g., multiple authors) or are used \emph{very}
rarely (e.g., appendices). Refer to the original ACM document for more
of those fancy examples.
% The current paper format *only* allows inline comments using the todo
% macro. That's kind of a bummer, and it would be neat if someone figured
% out how to change the acmconf style to allow this. I suspect it isn't *hard*
% but there are quite a few details that have to be sorted out in synchrony.
\todo[inline]{Needs more work}
\end{abstract}

\keywords{ACM proceedings, \LaTeX, text tagging}

\section{Introduction}
\label{sec:introduction}

The \textit{proceedings} are the records of a conference, and conference
editors like ACM seek to give their conference by-products a uniform,
high-quality appearance. We also would like our proceedings to look highly
professional, so we're borrowing heavily from the ACM formatting guidelines.
These include some
rigid requirements for the format of the proceedings documents: there
is a specified format (balanced  double columns), a specified
set of fonts (Arial or Helvetica and Times Roman) in
certain specified sizes (for instance, 9 point for body copy),
a specified live area (18 $\times$ 23.5 cm [7" $\times$ 9.25"]) centered on
the page, specified size of margins (2.54cm [1"] top and
bottom and 1.9cm [.75"] left and right; specified column width
(8.45cm [3.33"]) and gutter size (.083cm [.33"]).

\todo[inline, color=green]{I really like this section.}

The good news is, with only a handful of manual
settings,\footnote{One of these, the \texttt{\textbackslash alignauthor} command, 
you have
already used; another, \texttt{\textbackslash balancecolumns} will
be used in your very last run of \LaTeX\ to ensure
balanced column heights on the last page.} the \LaTeX\ document
class file handles all of this for you.

The remainder of this document is concerned with showing, in
the context of an ``actual'' document, the \LaTeX\ commands
specifically available for denoting the structure of a
proceedings paper, rather than with giving rigorous descriptions
or explanations of such commands. Section~\ref{sec:body} introduces the main
examples of formatting, and Section~\ref{sec:conclusions} wraps things up. 

\section{The {\secit Body} of The Paper}
\label{sec:body}

Typically, the body of a paper is organized
into a hierarchical structure, with numbered or unnumbered
headings for sections, subsections, sub-subsections, and even
smaller sections.  The command \texttt{\textbackslash section} that
precedes this paragraph is part of such a
hierarchy.\footnote{This is the second footnote.  It
starts a series of three footnotes that add nothing
informational, but just give an idea of how footnotes work
and look. It is a wordy one, just so you see
how a longish one plays out.} \LaTeX\ handles the numbering
and placement of these headings for you, when you use
the appropriate heading commands around the titles
of the headings.  If you want a sub-subsection or
smaller part to be unnumbered in your output, simply append an
asterisk to the command name.  Examples of both
numbered and unnumbered headings will appear throughout the
balance of this sample document.

Because the entire article is contained in
the \textbf{document} environment, you can indicate the
start of a new paragraph with a blank line in your
input file; that is why this sentence forms a separate paragraph.

\subsection{Type Changes and {\subsecit Special} Characters}
\label{sec:typeChangesSpecialChars}

We have already seen several typeface changes in this sample.  You
can indicate italicized words or phrases in your text with
the command \texttt{\textbackslash textit}; emboldening with the
command \texttt{\textbackslash textbf}
and typewriter-style (for instance, for computer code) with
\texttt{\textbackslash texttt}.
As a rule you'd prefer \texttt{\textbackslash emph} (which stands for \emph{emphasize})
over something like \texttt{\textbackslash textit} since that gives the typesetting system
more flexibility in how it can emphasize that text.

You do not
have to indicate typestyle changes when such changes are
part of the \textit{structural} elements of your
article; for instance, the heading of this subsection will
be in a sans serif\footnote{A third footnote, here.
Let's make this a rather short one to
see how it looks.} typeface, but that is handled by the
document class file. Take care with the use
of\footnote{A fourth, and last, footnote.}
the curly braces in typeface changes; they mark
the beginning and end of
the text that is to be in the different typeface.

You can use whatever symbols, accented characters, or
non-English characters you need anywhere in your document;
you can find a complete list of what is
available in the \textit{\LaTeX\
User's Guide}~\cite{Lamport:LaTeX}.

\subsection{Math Equations}
\label{sec:mathEquations}

You may want to display math equations in three distinct styles:
inline, numbered or non-numbered display.  Each of
the three are discussed in the next sections.

\subsubsection{Inline (In-text) Equations}
\label{sec:inlineEquations}

A formula that appears in the running text is called an
inline or in-text formula.  It is produced by the
\textbf{math} environment, which can be
invoked with the usual \texttt{\textbackslash begin\ldots\textbackslash end}
construction or with the short form \texttt{\$\ldots \$}. You
can use any of the symbols and structures,
from $\alpha$ to $\omega$, available in
\LaTeX~\cite{Lamport:LaTeX}; this section will simply show a
few examples of in-text equations in context. Notice how
this equation: \begin{math}\lim_{n\rightarrow \infty}x=0\end{math},
set here in in-line math style, looks slightly different when
set in display style.  (See next section).

\subsubsection{Display Equations}
\label{sec:displayEquations}

A numbered display equation -- one set off by vertical space
from the text and centered horizontally -- is produced
by the \textbf{equation} environment. An unnumbered display
equation is produced by the \textbf{displaymath} environment.

Again, in either environment, you can use any of the symbols
and structures available in \LaTeX; this section will just
give a couple of examples of display equations in context.
First, consider the equation, shown as an inline equation above:

\begin{equation*}
\lim_{n\rightarrow \infty}x=0
\end{equation*}

Notice how it is formatted somewhat differently in
the \textbf{displaymath}
environment.  Now, we'll enter an unnumbered equation:

\begin{displaymath}
	\sum_{i=0}^{\infty} x + 1
\end{displaymath}

and follow it with a numbered equation (Equation~\ref{eq:summation}):

\begin{equation}
	\sum_{i=0}^{\infty}x_i=\int_{0}^{\pi+2} f
\label{eq:summation}
\end{equation}

just to demonstrate \LaTeX's able handling of numbering.
Note that if you use numbered equations, you can give them \texttt{label}s
and \texttt{\textbackslash ref} them, e.g., Equation~\ref{eq:summation}.

\subsection{Multi-line formulas}
\label{sec:multiLineFormulas}

\subsection{Citations}
Citations to articles~\cite{Aaronson:2005,Garey:1979,Brun:2008} listed
in the Bibliography section of your
article will occur throughout the text of your article.
You should use BibTeX to automatically produce this bibliography;
you simply need to insert one of several citation commands with
a key of the item cited in the proper location in
the \texttt{.tex} file~\cite{OM:2008}.
The key is a short reference you invent to uniquely
identify each work; in this sample document, the key is
the first author's surname and a
word from the title.  This identifying key is included
with each item in the \texttt{.bib} file for your article.
The \texttt{align} and \texttt{align*} environments let you align sequences of
equations so, for examples, a sequences of equal signs line up nicely. 

% The & creates a "virtual tab stop" and corresponding &'s on each line
% will be aligned when the layout is done.
\begin{align*}
n_1 &= \sum_{i = 1}^k a_i \\
n_{x-y} &= \prod_{i = 1}^k b_i
\end{align*}

The details of the construction of the \texttt{.bib} file
are beyond the scope of this sample document, but more
information can be found in the \textit{Author's Guide},
and exhaustive details in the \textit{\LaTeX\ User's
Guide}.
\begin{align*}
 f(x) &= (x+a)(x+b) \\
 &= x^2 + (a+b)x + ab
\end{align*}


\subsection{Tables}
\label{sec:tables}

Because tables cannot be split across pages, the best
placement for them is typically the top of the page
nearest their initial cite.  To
ensure this proper ``floating'' placement of tables, use the
environment \textbf{table} to enclose the table's contents and
the table caption.  The contents of the table itself must go
in the \textbf{tabular} environment, to
be aligned properly in rows and columns, with the desired
horizontal and vertical rules.  Again, detailed instructions
on \textbf{tabular} material
is found in the \textit{\LaTeX\ User's Guide}.

Immediately following this sentence is the point at which
Table~\ref{tab:frequencyOfSpecialChars} is included in the input file; 
compare the placement of the table here with the table in the
PDF output running \LaTeX\ on this document.

\begin{table}[t]
\centering
\caption{Frequency of Special Characters}
\label{tab:frequencyOfSpecialChars}
\begin{tabular}{c|c|l}
Non-English or Math&Frequency & Comments\\ \hline
\O & 1 in 1,000 & For Swedish names\\
$\pi$ & 1 in 5 & Common in math\\
\$ & 4 in 5 & Used in business\\
$\Psi^2_1$ & 1 in 40,000 & Unexplained usage\\
\end{tabular}
\end{table}

To set a wider table, which takes up the whole width of
the page's live area, use the environment
\textbf{table*} to enclose the table's contents and
the table caption, as demonstrated in Table~\ref{tab:typicalCommands} below.  
As with a single-column table, this wide
table will ``float" to a location deemed more desirable.
Immediately following this sentence is the point at which
Table~\ref{tab:typicalCommands} is included in the input file; again, it is
instructive to compare the placement of the
table here with the table in the printed dvi
output of this document.


\begin{table*}[t]
\centering
\caption{Some Typical Commands}
\label{tab:typicalCommands}
\begin{tabular}{ccl}
Command & A Number & Comments \\ \hline
\texttt{\textbackslash alignauthor}     & 100 & Author alignment\\
\texttt{\textbackslash numberofauthors} & 200 & Author enumeration\\
\texttt{\textbackslash table}           & 300 & For tables\\
\texttt{\textbackslash table*}          & 400 & For wider tables\\ 
\end{tabular}
\end{table*}
% end the environment with {table*}, NOTE not {table}!

\subsection{Citations}
\label{sec:citations}

Citations to articles~\cite{Aaronson:2005,Garey:1979,Brun:2008} listed
in the Bibliography section of your
article will occur throughout the text of your article.
You should use BibTeX to automatically produce this bibliography;
you simply need to insert one of several citation commands with
a key of the item cited in the proper location in
the \texttt{.tex} file~\cite{OM:2008}.
The key is a short reference you invent to uniquely
identify each work; in this sample document, the key is
the first author's surname and a
word from the title.  This identifying key is included
with each item in the \texttt{.bib} file for your article.

It is recommended that you precede \texttt{\textbackslash cite} (and 
\texttt{\textbackslash ref}) commands with a tilde
character instead of a space, e.g., \texttt{some text\textasciitilde\textbackslash cite}. The tilde gives you a non-breaking space which ensures that your citation won't get
stranded by itself on the beginning of a line.

The details of the construction of the \texttt{.bib} file
are beyond the scope of this sample document, but more
information can be found in the \textit{Author's Guide},
and exhaustive details in the \textit{\LaTeX\ User's
Guide}.

This article shows only the plainest form
of the citation command, using \texttt{\textbackslash cite},
which is all that is needed for our senior seminar.
You shouldn't use any other forms here.

\subsection{Theorem-like Constructs}
\label{sec:theoremLikeConstructs}

Other common constructs that may occur in your article are
the forms for logical constructs like theorems, axioms,
corollaries and proofs.  There are
two forms, one produced by the
command \texttt{\textbackslash newtheorem} and the
other by the command \texttt{\textbackslash newdef}; perhaps
the clearest and easiest way to distinguish them is
to compare the two in the output of this sample document:

Theorem~\ref{thm:integration} below uses the \textbf{theorem} environment, created by
the \texttt{\textbackslash newtheorem} command:

% You would usually put a \newtheorem command up at the top 
% of your LaTeX document after the \usepackage commands. It's
% just here in this example so it's with the text that describes it.
\newtheorem{theorem}{Theorem}

\begin{theorem}
Let $f$ be continuous on $[a,b]$.  If $G$ is
an antiderivative for $f$ on $[a,b]$, then
\begin{displaymath}\int^b_af(t)dt = G(b) - G(a).\end{displaymath}
\label{thm:integration}
\end{theorem}

The other uses the \textbf{definition} environment, created
by the \texttt{\textbackslash newdef} command:
\newdef{definition}{Definition}
\begin{definition}
If $z$ is irrational, then by $e^z$ we mean the
unique number which has
logarithm $z$: \begin{displaymath}{\log e^z = z}\end{displaymath}
\end{definition}

Two lists of constructs that use one of these
forms is given in the
\textit{Author's  Guidelines}.
 
There is one other similar construct environment, which is
already set up
for you; i.e. you must \textit{not} use
a \texttt{\textbackslash newdef} command to
create it: the \textbf{proof} environment.  Here
is a example of its use:
\begin{proof}
Suppose on the contrary there exists a real number $L$ such that
\begin{displaymath}
\lim_{x\rightarrow\infty} \frac{f(x)}{g(x)} = L.
\end{displaymath}
Then
\begin{align*}
l &= \lim_{x\rightarrow c} f(x) \\
  &= \lim_{x\rightarrow c}
\left[ g{x} \cdot \frac{f(x)}{g(x)} \right ] \\
  &= \lim_{x\rightarrow c} g(x) \cdot \lim_{x\rightarrow c}
\frac{f(x)}{g(x)}  \\
  &= 0\cdot L  \\
  &= 0,
\end{align*}
which contradicts our assumption that $l\neq 0$.
\end{proof}

Complete rules about using these environments and using the
two different creation commands are in the
\textit{Author's Guide}; please consult it for more
detailed instructions.  If you need to use another construct,
not listed therein, which you want to have the same
formatting as the Theorem
or the Definition~\cite{salas:calculus} shown above,
use the \texttt{\textbackslash newtheorem} or the
\texttt{\textbackslash newdef} command,
respectively, to create it.

\subsection*{A {\secit Caveat} for the \TeX\ Expert}
\label{sec:caveatForExperts}

Because you have just been given permission to
use the \texttt{\textbackslash newdef} command to create a
new form, you might think you can
use \TeX's \texttt{\textbackslash def} to create a
new command: \textit{Please refrain from doing this!}
Remember that your \LaTeX\ source code is primarily intended
to create camera-ready copy, but may be converted
to other forms -- e.g. HTML. If you inadvertently omit
some or all of the \texttt{\textbackslash def}s recompilation will
be, to say the least, problematic.

\section{Conclusions}
\label{sec:conclusions}

This paragraph will end the body of this sample document.
Remember that you might still have Acknowledgments or
Appendices; brief samples of these
follow.  There is still the Bibliography to deal with; and
we will make a disclaimer about that here: with the exception
of the reference to the \LaTeX\ book, the citations in
this paper are to articles which have nothing to
do with the present subject and are used as
examples only.

\section*{Acknowledgments}
\label{sec:acknowledgments}

This section is optional; it is a location for you
to acknowledge grants, funding, editing assistance and
what have you.

It is common (but by no means necessary) for students to thank
their advisor, and possibly other faculty, friends, and family who provided
useful feedback on the paper as it was being written.

In the present case, for example, the
authors would like to thank Gerald Murray of ACM for
his help in codifying this \textit{Author's Guide}
and the \textbf{.cls} and \textbf{.tex} files that it describes.

% The following two commands are all you need in the
% initial runs of your .tex file to
% produce the bibliography for the citations in your paper.
\bibliographystyle{abbrv}
% sample_paper.bib is the name of the BibTex file containing the
% bibliography entries. Note that you *don't* include the .bib ending here.
\bibliography{sample_paper}  
% You must have a proper ".bib" file
%  and remember to run:
% latex bibtex latex latex
% to resolve all references

\end{document}
